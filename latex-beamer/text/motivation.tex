%!TEX TS-program = ../make.zsh

\begin{frame}[fragile]{Motivation and Scope}

  \begin{columns}
    \begin{column}{0.6\textwidth}
      \centering\begin{tikzpicture}[scale=0.8, small mindmap,
        %root concept/.append style={grow=0},
        %level 1 concept/.append style={grow=0},
        %level 1 concept/.append style={level distance=130,sibling angle=30},
        %extra concept/.append style={color=blue!50,text=black},
        %concept/.append style={sibling distance=10mm, grow=0},
        %level 1 concept/.append style={sibling distance=40},
        %level 2 concept/.append style={sibling distance=40},
        group/.style={concept, concept color=green!50},
        user/.style={concept, concept color=yellow!30},
        page/.style={concept, concept color=blue!40}]

        \node [group] {Jeder}
          [sibling distance=100]
          child {node [group] (alle mitglieder) {Alle Mitglieder}}
          child {node [group] {Orts-Gruppen}
            child {node [group] {Berlin}}
            child {node [group] {Halle}}
            child {node [group] {Nürnberg}
              child {node [group] (n studenten) {Studenten}
                child {node [page] {Protokolle}}
                child {node [user] (max) {Max}}
                child {node [user] (lena) {Lena}}
                child {node [group] {Amtsträger}
                  child {node [group] (n studenten schriftwart) {Schriftwart}}
                }
              }
              child {node [group] (n alumni) {Alumni}
                child {node [page] {Protokolle}}
                child {node [user] (hans) {Hans}}
                child {node [user] (gretel) {Gretel}}
                child {node [group] {Amtsträger}
                  child {node [group] (n alumni schriftwart) {Schriftwart}}
                }
              }
              child {node [group] {Gäste}
                child {node [user] {Peter}}
              }
            }
          }
          child {node [group] {Alle Amtsträger}
            child {node [group] (alle schriftwarte) {Alle Schriftwarte}}
          }
        ;

        \begin{pgfonlayer}{background}
          \path (n studenten schriftwart) to [circle connection bar] (lena);
          \path (n alumni schriftwart) to [circle connection bar] (gretel);
          \path (alle schriftwarte) to [circle connection bar] (n alumni schriftwart);
          \path (alle schriftwarte) to [circle connection bar] (n studenten schriftwart);

          \path (alle mitglieder) to [circle connection bar] (n studenten);
          \path (alle mitglieder) to [circle connection bar] (n alumni);
        \end{pgfonlayer}
      \end{tikzpicture}
    \end{column}

    \begin{column}{0.4\textwidth}
      \begin{itemize}
        \item Sometimes we need to model data as graphs: nodes with connections.
        \item Queries should be expressive: \inlineruby{Group.first.descendant_groups}.
        \item Compare several approaches how to do this with ruby.
      \end{itemize}
    \end{column}
  \end{columns}

\end{frame}

