%!TEX TS-program = ../make.zsh

\begin{frame}[fragile]{Motivation and Scope}

  \begin{columns}
    \begin{column}{0.6\textwidth}
      \scalebox{0.5}{
        \centering\begin{tikzpicture}[
            group/.style={circle, draw, fill=green!50},
            user/.style={circle, draw, fill=yellow!30},
            page/.style={circle, draw, fill=blue!40},
            link/.style={circle connection bar, fill},
            bglink/.style={link, fill=black!10},
          ]

          \graph [
              tree layout,
              nodes={circle},
              edges={link},
              grow=down,
              level distance=0.5in,
              sibling distance=0.5in
              ] {
            Jeder [group] -> Alle Mitglieder [group];
            Jeder -> Ortsgruppen [group];
            Jeder -> Alle Amtsträger [group];

            Ortsgruppen -> {
              Berlin [group],
              Halle [group],
              Nürnberg [group]
            };

            Nürnberg -> {
              Studenten [group],
              Alumni [group],
              Gäste [group]
            };

            Studenten -> {
              Studenten Protokolle/"Protokolle" [page],
              Max [user],
              Lena [user],
              Nürnberg Studenten Amtsträger/"Amtsträger" [group] -> {
                Nürnberg Studenten Schriftwart/"Schriftwart" [group]
              }
            };

            Nürnberg Studenten Schriftwart -> Lena;

            Alumni -> {
              Alumni Protokolle/"Protokolle" [page],
              Hans [user],
              Gretel [user],
              Nürnberg Alumni Amtsträger/"Amtsträger" [group] -> {
                Nürnberg Alumni Schriftwart/"Schriftwart" [group]
              }
            };

            Nürnberg Alumni Schriftwart -> Gretel;

            Gäste -> Peter [user];

            Alle Amtsträger -> Alle Schriftwarte [group] -> Leitfaden [page];

            Alle Mitglieder -> Verbandszeitschrift [page] -> Ausgabe 1 [page];
          };

          \begin{pgfonlayer}{background}
            \path (Alle Schriftwarte) to [bglink] (Nürnberg Studenten Schriftwart);
            \path (Alle Schriftwarte) to [bglink] (Nürnberg Alumni Schriftwart);

            \path (Alle Mitglieder) to [bglink] (Studenten);
            \path (Alle Mitglieder) to [bglink] (Alumni);
          \end{pgfonlayer}

          \draw (Max) edge [link, color=red] node [fill=white, above left, draw=red] {valid\_to: 2009} (Studenten);
          \draw (Max) edge [link, color=red] node [fill=white, above left, draw=red] {valid\_from: 2009} (Alumni);

        \end{tikzpicture}
      }
    \end{column}

    \begin{column}{0.4\textwidth}
      \begin{itemize}
        \item Sometimes we need to model data as graphs: nodes with connections.
        \item Queries should be expressive: \inlineruby{Group.first.descendant_groups}.
        \item Compare several approaches how to do this with ruby.
      \end{itemize}
    \end{column}
  \end{columns}

\end{frame}

