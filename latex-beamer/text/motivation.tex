%!TEX TS-program = ../make.zsh

\newcommand\samplegraph[1]{
  \directlua{
    function apply_styles (styles)
      for style in string.gmatch(styles, "hide[^,]*") do
        style_to_hide = string.gsub(style, "hide", "")
        tex.sprint("\\tikzset{" .. style_to_hide .. "/.style={opacity=0, draw=none, text=white, fill=white!0}}")
      end
    end
  }

  \centering\begin{tikzpicture}[
      every node/.style={circle, draw},
      group/.style={fill=green!50},
      user/.style={fill=yellow!30},
      page/.style={fill=blue!40},
      link/.style={circle connection bar, fill},
      bglink/.style={link, fill=black!10},
    ]
    \directlua{apply_styles("#1")}

    \graph [
        tree layout,
        nodes={circle},
        edges={link},
        grow=down,
        level distance=0.5in,
        sibling distance=0.5in
        ] {
      Jeder [group] -> [supergroupstructure] Alle Mitglieder [group,supergroupstructure];
      Jeder -> Ortsgruppen [group];
      Jeder -> [supergroupstructure] Alle Amtsträger [group,supergroupstructure];

      Ortsgruppen -> {
        Berlin [group],
        Halle [group],
        Nürnberg [group]
      };

      Nürnberg -> {
        Studenten [group],
        Alumni [group],
        Gäste [group]
      };

      Studenten -> {
        Studenten Protokolle/"Protokolle" [page],
        Max [user],
        Lena [user],
        Nürnberg Studenten Amtsträger/"Amtsträger" [group] -> {
          Nürnberg Studenten Schriftwart/"Schriftwart" [group]
        }
      };

      Nürnberg Studenten Schriftwart -> Lena;

      Alumni -> {
        Alumni Protokolle/"Protokolle" [page],
        Hans [user],
        Gretel [user],
        Nürnberg Alumni Amtsträger/"Amtsträger" [group] -> {
          Nürnberg Alumni Schriftwart/"Schriftwart" [group]
        }
      };

      Nürnberg Alumni Schriftwart -> Gretel;

      Gäste -> Peter [user];

      Alle Amtsträger -> [supergroupstructure] Alle Schriftwarte [group,supergroupstructure] -> [supergroupstructure] Leitfaden [page,supergroupstructure];

      Alle Mitglieder -> [supergroupstructure] Verbandszeitschrift [page,supergroupstructure] -> [supergroupstructure] Ausgabe 1 [page,supergroupstructure];
    };

    \begin{pgfonlayer}{background}
      \path (Alle Schriftwarte) to [bglink,supergroupstructure] (Nürnberg Studenten Schriftwart);
      \path (Alle Schriftwarte) to [bglink,supergroupstructure] (Nürnberg Alumni Schriftwart);

      \path (Alle Mitglieder) to [bglink,supergroupstructure] (Studenten);
      \path (Alle Mitglieder) to [bglink,supergroupstructure] (Alumni);
    \end{pgfonlayer}

    \path (Max) to [link, edge label={node [validityrange] {Foo}}] (Studenten);
    \path (Max) to [bglink, edge label={node [validityrange] {Bar}}] (Alumni);

  \end{tikzpicture}
}

\begin{frame}[fragile]{Motivation and Scope}

  \begin{columns}
    \begin{column}{0.6\textwidth}
      \scalebox{0.5}{
        \only<1>{\samplegraph{hidesupergroupstructure}}
        \only<2>{\samplegraph{}}
      }
    \end{column}

    \begin{column}{0.4\textwidth}
      \begin{itemize}
        \item Sometimes we need to model data as graphs: nodes with connections.
        \item Queries should be expressive: \inlineruby{Group.first.descendant_groups}.
        \item Compare several approaches how to do this with ruby.
      \end{itemize}
    \end{column}
  \end{columns}

\end{frame}

